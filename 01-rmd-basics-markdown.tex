%%%%%%%%%%%%%%%%%%%%%%%%%%%%%%%%%%%%%%%%%%%%%%%%%%%%%%%%%%%%%%%
%% BRIEF VERSION OF OXFORD THESIS TEMPLATE FOR CHAPTER PREVIEWS

%%%%% CHOOSE PAGE LAYOUT
% format for PDF output (ie equal margins, no extra blank pages):
\documentclass[a4paper,nobind]{templates/ociamthesis}

% UL 30 Nov 2018 pandoc puts lists in 'tightlist' command when no space between bullet points in Rmd file
\providecommand{\tightlist}{%
  \setlength{\itemsep}{0pt}\setlength{\parskip}{0pt}}
  
% UL 1 Dec 2018, fix to include code in shaded environments
\usepackage{color}
\usepackage{fancyvrb}
\newcommand{\VerbBar}{|}
\newcommand{\VERB}{\Verb[commandchars=\\\{\}]}
\DefineVerbatimEnvironment{Highlighting}{Verbatim}{commandchars=\\\{\}}
% Add ',fontsize=\small' for more characters per line
\usepackage{framed}
\definecolor{shadecolor}{RGB}{248,248,248}
\newenvironment{Shaded}{\begin{snugshade}}{\end{snugshade}}
\newcommand{\AlertTok}[1]{\textcolor[rgb]{0.94,0.16,0.16}{#1}}
\newcommand{\AnnotationTok}[1]{\textcolor[rgb]{0.56,0.35,0.01}{\textbf{\textit{#1}}}}
\newcommand{\AttributeTok}[1]{\textcolor[rgb]{0.77,0.63,0.00}{#1}}
\newcommand{\BaseNTok}[1]{\textcolor[rgb]{0.00,0.00,0.81}{#1}}
\newcommand{\BuiltInTok}[1]{#1}
\newcommand{\CharTok}[1]{\textcolor[rgb]{0.31,0.60,0.02}{#1}}
\newcommand{\CommentTok}[1]{\textcolor[rgb]{0.56,0.35,0.01}{\textit{#1}}}
\newcommand{\CommentVarTok}[1]{\textcolor[rgb]{0.56,0.35,0.01}{\textbf{\textit{#1}}}}
\newcommand{\ConstantTok}[1]{\textcolor[rgb]{0.00,0.00,0.00}{#1}}
\newcommand{\ControlFlowTok}[1]{\textcolor[rgb]{0.13,0.29,0.53}{\textbf{#1}}}
\newcommand{\DataTypeTok}[1]{\textcolor[rgb]{0.13,0.29,0.53}{#1}}
\newcommand{\DecValTok}[1]{\textcolor[rgb]{0.00,0.00,0.81}{#1}}
\newcommand{\DocumentationTok}[1]{\textcolor[rgb]{0.56,0.35,0.01}{\textbf{\textit{#1}}}}
\newcommand{\ErrorTok}[1]{\textcolor[rgb]{0.64,0.00,0.00}{\textbf{#1}}}
\newcommand{\ExtensionTok}[1]{#1}
\newcommand{\FloatTok}[1]{\textcolor[rgb]{0.00,0.00,0.81}{#1}}
\newcommand{\FunctionTok}[1]{\textcolor[rgb]{0.00,0.00,0.00}{#1}}
\newcommand{\ImportTok}[1]{#1}
\newcommand{\InformationTok}[1]{\textcolor[rgb]{0.56,0.35,0.01}{\textbf{\textit{#1}}}}
\newcommand{\KeywordTok}[1]{\textcolor[rgb]{0.13,0.29,0.53}{\textbf{#1}}}
\newcommand{\NormalTok}[1]{#1}
\newcommand{\OperatorTok}[1]{\textcolor[rgb]{0.81,0.36,0.00}{\textbf{#1}}}
\newcommand{\OtherTok}[1]{\textcolor[rgb]{0.56,0.35,0.01}{#1}}
\newcommand{\PreprocessorTok}[1]{\textcolor[rgb]{0.56,0.35,0.01}{\textit{#1}}}
\newcommand{\RegionMarkerTok}[1]{#1}
\newcommand{\SpecialCharTok}[1]{\textcolor[rgb]{0.00,0.00,0.00}{#1}}
\newcommand{\SpecialStringTok}[1]{\textcolor[rgb]{0.31,0.60,0.02}{#1}}
\newcommand{\StringTok}[1]{\textcolor[rgb]{0.31,0.60,0.02}{#1}}
\newcommand{\VariableTok}[1]{\textcolor[rgb]{0.00,0.00,0.00}{#1}}
\newcommand{\VerbatimStringTok}[1]{\textcolor[rgb]{0.31,0.60,0.02}{#1}}
\newcommand{\WarningTok}[1]{\textcolor[rgb]{0.56,0.35,0.01}{\textbf{\textit{#1}}}}

%UL 2 Dec 2018 add a bit of white space before and after code blocks
\renewenvironment{Shaded}
{
  \vspace{4pt}%
  \begin{snugshade}%
}{%
  \end{snugshade}%
  \vspace{4pt}%
}
%UL 2 Dec 2018 reduce whitespace around verbatim environments
\usepackage{etoolbox}
\makeatletter
\preto{\@verbatim}{\topsep=0pt \partopsep=0pt }
\makeatother

%UL 28 Mar 2019, enable strikethrough
\usepackage[normalem]{ulem}

%UL 3 Nov 2019, avoid mysterious error from not having hyperref included
\usepackage{hyperref}

%%%%% SELECT YOUR DRAFT OPTIONS
% Three options going on here; use in any combination.  But remember to turn the first two off before
% generating a PDF to send to the printer!

% This adds a "DRAFT" footer to every normal page.  (The first page of each chapter is not a "normal" page.)

% This highlights (in blue) corrections marked with (for words) \mccorrect{blah} or (for whole
% paragraphs) \begin{mccorrection} . . . \end{mccorrection}.  This can be useful for sending a PDF of
% your corrected thesis to your examiners for review.  Turn it off, and the blue disappears.

%%%%% BIBLIOGRAPHY SETUP
% Note that your bibliography will require some tweaking depending on your department, preferred format, etc.
% The options included below are just very basic "sciencey" and "humanitiesey" options to get started.
% If you've not used LaTeX before, I recommend reading a little about biblatex/biber and getting started with it.
% If you're already a LaTeX pro and are used to natbib or something, modify as necessary.
% Either way, you'll have to choose and configure an appropriate bibliography format...

% The science-type option: numerical in-text citation with references in order of appearance.
% \usepackage[style=numeric-comp, sorting=none, backend=biber, doi=false, isbn=false]{biblatex}
% \newcommand*{\bibtitle}{References}

% The humanities-type option: author-year in-text citation with an alphabetical works cited.
% \usepackage[style=authoryear, sorting=nyt, backend=biber, maxcitenames=2, useprefix, doi=false, isbn=false]{biblatex}
% \newcommand*{\bibtitle}{Works Cited}

%UL 3 Dec 2018: set this from YAML in index.Rmd
\usepackage[style=numeric-comp, sorting=none, backend=biber, doi=false, isbn=false]{biblatex}
\newcommand*{\bibtitle}{References}

% This makes the bibliography left-aligned (not 'justified') and slightly smaller font.
\renewcommand*{\bibfont}{\raggedright\small}

% Change this to the name of your .bib file (usually exported from a citation manager like Zotero or EndNote).
\addbibresource{references.bib}

%%%%% YOUR OWN PERSONAL MACROS
% This is a good place to dump your own LaTeX macros as they come up.

% To make text superscripts shortcuts
	\renewcommand{\th}{\textsuperscript{th}} % ex: I won 4\th place
	\newcommand{\nd}{\textsuperscript{nd}}
	\renewcommand{\st}{\textsuperscript{st}}
	\newcommand{\rd}{\textsuperscript{rd}}

%%%%% THE ACTUAL DOCUMENT STARTS HERE
\begin{document}

%%%%% CHOOSE YOUR LINE SPACING HERE
% This is the official option.  Use it for your submission copy and library copy:
\setlength{\textbaselineskip}{22pt plus2pt}
% This is closer spacing (about 1.5-spaced) that you might prefer for your personal copies:
%\setlength{\textbaselineskip}{18pt plus2pt minus1pt}

% UL: You can set the general paragraph spacing here - I've set it to 2pt (was 0) so
% it's less claustrophobic
\setlength{\parskip}{2pt plus 1pt}

% Leave this line alone; it gets things started for the real document.
\setlength{\baselineskip}{\textbaselineskip}

% all your chapters and appendices will appear here
\begin{savequote}
Neque porro quisquam est qui dolorem ipsum quia dolor sit amet,
consectetur, adipisci velit\ldots{}

There is no one who loves pain itself, who seeks after it and wants to
have it, simply because it is pain\ldots{}
\end{savequote}



\hypertarget{Chpt1}{%
\chapter{Binaural listening: interripted and alternated speech-in-noise in adults}\label{Chpt1}}

\minitoc 

\hypertarget{influence-of-distractor-type-on-im}{%
\section{Influence of distractor type on IM}\label{influence-of-distractor-type-on-im}}

\hypertarget{introduction}{%
\subsection{Introduction}\label{introduction}}

Communication in adverse listening situations where the target speech is incomplete or distorted is a typical everyday occurrence. Often, the sound source of interest is masked by nearby interfering sounds (e.g., traffic noise or competing talkers) or degraded (e.g., due to reverberations, transmission artefacts or filtering). Remarkably however, listeners can often maintain high speech intelligibility even when large portions of the speech signal are physically missing or entirely masked by other sounds (Miller and Licklider 1950; Başkent et al. 2016). This phenomenon is, among other things, attributed to the redundant characteristics of speech in the spectral and the temporal domain, enabling the listener to piece together short glimpses of the target signal to achieve high speech perception (i.e., {``glimpsing theory''}; Cooke 2006).
The way our auditory system overcomes such impoverished listening conditions is not well understood.
One of the main obstacles when trying to answer this question is the large variation in performance across listeners, in particular in more ecological listening scenarios with several competing talkers with different complex spectro-temporal properties (Surprenant and Watson 2001).
In many cases, such individual differences cannot be explained by hearing sensitivity as measured with pure-tone-audiogram (L. E. Humes and Dubno 2010; Kidd and Humes 2012).
Individual differences may arise from variations in the listeners' auditory processing abilities or their abilities to make use of perceptual acoustic and linguistic information (Pichora-Fuller and Singh 2006; Surprenant and Watson 2001).
In addition, there is an increasing amount of evidence suggesting that variability in speech perception may be in part attributed to variations in cognitive abilities, especially in adverse listening conditions where the distractor is speech or speech-like (see review by Akeroyd 2008; Arlinger et al. 2009; Kidd and Humes 2012; Esch et al. 2013; Larry E. Humes, Kidd, and Lentz 2013).
Understanding what causes certain groups of listeners to experience listening difficulties under challenging listening situations can help us finding better intervention plans or treatments that fit to their individual needs. Moreover, we can use this knowledge to improve currently used speech recognition and speech enhancement techniques.
However, isolating and quantifying the contribution of the different mechanisms involved throughout the auditory system is challenging.\\

The present paper aims to investigate the utility of a novel speech-on-speech listening task that appears to demand higher-level cognitive aspects of listening and may aid in disentangling the reasons why different groups of people experience difficulty in listening in noisy situations. In the task, target speech is interrupted and segmented at a fixed rate. The segments are then alternated between the two ears out-of-phase with an interrupted distractor which is alternated in a similar way, resulting in alternated segments of both signals between the two ears, with only one stimulus present in each ear at any given time. The task necessitates the listeners' ability to switch and sustain their attention on the target speech, while inhibiting the distractor segments, and to integrate the short-term auditory information between the two ears.
A preliminary study (unpublished BSc thesis Akinseye 2015) compared performance in the task across young (mean age: 24, range: 20 - 33 years old) and older adults (mean age: 63, range: 50-72 years old) with audiometrically normal hearing up to 4\textasciitilde kHz. Normal cognitive skills were controlled for the older group using a standard screening test (MoCA; Nasreddine et al. 2005). Interestingly, while no significant difference in speech intelligibility was found between the young and older adults for a ``standard" speech-in-noise test, there was a highly significant difference in performance between the groups, with older adults showing poorer intelligibility for the switching task when presented with connected speech as a distractor. These results suggest that the switching task may demand some higher-level cognitive aspects of listening that are not probed by more simple listening tasks.
The objective here is to investigate different aspects of the task across normal hearing young adults. This includes examining the effect of distractor types (speech vs.~non-speech); intelligibility of the speech distractors; and similarity between the target and the distractor, for same- and opposite-sex distractor talker configurations on the listeners' speech perception. In addition, test-retest reliability and reproducibility of the task's score is evaluated.
To set the context, it is beneficial to review some aspects involved in speech perception in a `Cocktail-party'-like environment (Cherry 1953) as an effect of distractor interference, interruption, and alternation.

\hypertarget{Exp1}{%
\subsection{Experiment I: speech vs.~non-speech distractors}\label{Exp1}}

\hypertarget{methods}{%
\subsubsection{Methods}\label{methods}}

\hypertarget{participants}{%
\paragraph{Participants}\label{participants}}

\hypertarget{stimuli}{%
\paragraph{Stimuli}\label{stimuli}}

\hypertarget{the-switching-task}{%
\paragraph{The switching task}\label{the-switching-task}}

\hypertarget{procedure}{%
\paragraph{Procedure}\label{procedure}}

\hypertarget{statistical-methods-exp1-stats}{%
\paragraph{Statistical methods \{Exp1-Stats\}}\label{statistical-methods-exp1-stats}}

\hypertarget{results}{%
\subsubsection{Results}\label{results}}

\hypertarget{discussion}{%
\subsubsection{Discussion}\label{discussion}}

\hypertarget{experiment-ii-speech-distractors-spoken-in-a-familiar-vs.-unfamiliar-language}{%
\subsection{Experiment II: speech distractors spoken in a familiar vs.~unfamiliar language}\label{experiment-ii-speech-distractors-spoken-in-a-familiar-vs.-unfamiliar-language}}

Findings in the first experiment demonstrated that performance in the task is specifically affected when speech distractors are used, and that this IM effect did not occur for the non-speech distractors. To extend these findings, in the second experiment we examined the contributions to IM of familiarity with the spoken language of the distractor (English vs.~Mandarin), and similarity-related features as in voice characteristics of the talkers (same-sex vs.~opposite-sex talkers). Furthermore, the applicability of the proposed task for future clinical and research use was examined.

\hypertarget{methods-1}{%
\subsubsection{Methods}\label{methods-1}}

\hypertarget{participants-1}{%
\paragraph{Participants}\label{participants-1}}

\hypertarget{stimuli-1}{%
\paragraph{Stimuli}\label{stimuli-1}}

\hypertarget{procedure-1}{%
\paragraph{Procedure}\label{procedure-1}}

\hypertarget{results-1}{%
\subsubsection{Results}\label{results-1}}

\hypertarget{within-session-test-retest-reliability}{%
\paragraph{Within-session test-retest reliability}\label{within-session-test-retest-reliability}}

\hypertarget{score-reproducibility-a-comparison-between-experiment-i-and-ii}{%
\paragraph{Score reproducibility --- a comparison between experiment I and II}\label{score-reproducibility-a-comparison-between-experiment-i-and-ii}}

\hypertarget{effects-of-the-distractors-language-familiarity-and-talker-sex-on-im}{%
\paragraph{Effects of the distractor's language familiarity and talker-sex on IM}\label{effects-of-the-distractors-language-familiarity-and-talker-sex-on-im}}

\hypertarget{discussion-1}{%
\subsubsection{Discussion}\label{discussion-1}}

\hypertarget{within-session-test-retest-reliability-1}{%
\paragraph{Within-session test-retest reliability}\label{within-session-test-retest-reliability-1}}

\hypertarget{score-reproducibility-a-comparison-between-experiment-i-and-ii-1}{%
\paragraph{Score reproducibility --- a comparison between experiment I and II}\label{score-reproducibility-a-comparison-between-experiment-i-and-ii-1}}

\hypertarget{effects-of-distractors-language-familiarity-and-talker-sex-on-im}{%
\paragraph{Effects of distractor's language familiarity and talker sex on IM}\label{effects-of-distractors-language-familiarity-and-talker-sex-on-im}}

\hypertarget{general-discussion-and-conclusion}{%
\subsection{General discussion and conclusion}\label{general-discussion-and-conclusion}}

\hypertarget{dichotic-vs.-monotic-presentation-and-the-influence-of-speech-material}{%
\section{Dichotic vs.~monotic presentation and the influence of speech material}\label{dichotic-vs.-monotic-presentation-and-the-influence-of-speech-material}}

\hypertarget{introduction-1}{%
\subsection{Introduction}\label{introduction-1}}

\hypertarget{methods-2}{%
\subsection{Methods}\label{methods-2}}

\hypertarget{results-2}{%
\subsection{Results}\label{results-2}}

To create numbered equations, put them in an `equation' environment and give them a label with the syntax \texttt{(\textbackslash{}\#eq:label)}, like this:

\begin{Shaded}
\begin{Highlighting}[]
\KeywordTok{\textbackslash{}begin}\NormalTok{\{}\ExtensionTok{equation}\NormalTok{\}}\SpecialStringTok{ }
\SpecialStringTok{  f}\SpecialCharTok{\textbackslash{}left}\SpecialStringTok{(k}\SpecialCharTok{\textbackslash{}right}\SpecialStringTok{) = }\SpecialCharTok{\textbackslash{}binom}\SpecialStringTok{\{n\}\{k\} p\^{}k}\SpecialCharTok{\textbackslash{}left}\SpecialStringTok{(1{-}p}\SpecialCharTok{\textbackslash{}right}\SpecialStringTok{)\^{}\{n{-}k\}}
\SpecialStringTok{  (}\SpecialCharTok{\textbackslash{}\#}\SpecialStringTok{eq:binom)}
\KeywordTok{\textbackslash{}end}\NormalTok{\{}\ExtensionTok{equation}\NormalTok{\} }
\end{Highlighting}
\end{Shaded}

Becomes:
\begin{equation}
f\left(k\right)=\binom{n}{k}p^k\left(1-p\right)^{n-k}
\label{eq:binom}
\end{equation}

For more (e.g.~how to theorems), see e.g.~the documentation on \href{https://bookdown.org/yihui/bookdown/markdown-extensions-by-bookdown.html\#equations}{bookdown.org}

\hypertarget{discussion-2}{%
\subsection{Discussion}\label{discussion-2}}

\begin{itemize}
\item
  \emph{R Markdown: The Definitive Guide} - \url{https://bookdown.org/yihui/rmarkdown/}
\item
  \emph{R for Data Science} - \url{https://r4ds.had.co.nz}
\end{itemize}

\hypertarget{conclusion}{%
\subsection*{Conclusion}\label{conclusion}}
\addcontentsline{toc}{subsection}{Conclusion}

\hypertarget{refs}{}
\begin{CSLReferences}{1}{0}
\leavevmode\hypertarget{ref-Akeroyd2008}{}%
Akeroyd, Michael A. 2008. {``{Are individual differences in speech reception related to individual differences in cognitive ability? A survey of twenty experimental studies with normal and hearing-impaired adults}.''} \emph{International Journal of Audiology} 47 (SUPPL. 2).

\leavevmode\hypertarget{ref-Akinseye2015}{}%
Akinseye, Gladys. 2015. {``{The perception of interrupted and speech in older and younger adults with normal hearing.}''} Unpublished BSc thesis, University College London, UCL.

\leavevmode\hypertarget{ref-Arlinger2009}{}%
Arlinger, Stig, Thomas Lunner, Björn Lyxell, and M. Kathleen Pichora-Fuller. 2009. {``{The emergence of cognitive hearing science}.''} \emph{Scandinavian Journal of Psychology} 50 (5): 371--84. \url{https://doi.org/10.1111/j.1467-9450.2009.00753.x}.

\leavevmode\hypertarget{ref-Baskent2016}{}%
Başkent, Deniz, Jeanne Clarke, Carina Pals, Michel R Benard, Pranesh Bhargava, Jefta Saija, Anastasios Sarampalis, Anita Wagner, and Etienne Gaudrain. 2016. {``{Cognitive Compensation of Speech Perception With Hearing Impairment, Cochlear Implants, and Aging: How and to What Degree Can It Be Achieved?}''} \emph{Trends in Hearing} 20 (October). \url{https://doi.org/10.1177/2331216516670279}.

\leavevmode\hypertarget{ref-Cherry1953}{}%
Cherry, E. Colin. 1953. {``{Some Experiments on the Recognition of Speech, with One and with Two Ears}.''} \emph{The Journal of the Acoustical Society of America} 25 (5): 975--79. \url{https://doi.org/10.1121/1.1907229}.

\leavevmode\hypertarget{ref-Cooke2006}{}%
Cooke, Martin. 2006. {``{A glimpsing model of speech perception in noise}.''} \emph{The Journal of the Acoustical Society of America} 119 (3): 1562--73. \url{https://doi.org/10.1121/1.2166600}.

\leavevmode\hypertarget{ref-VanEsch2013}{}%
Esch, Thamar E. M. van, Birger Kollmeier, Matthias Vormann, Johannes Lyzenga, Tammo Houtgast, Mathias Hällgren, Birgitta Larsby, Sheetal P. Athalye, Mark E. Lutman, and Wouter A. Dreschler. 2013. {``{Evaluation of the preliminary auditory profile test battery in an international multi-centre study}.''} \emph{International Journal of Audiology} 52 (5): 305--21. \url{https://doi.org/10.3109/14992027.2012.759665}.

\leavevmode\hypertarget{ref-Humes2010}{}%
Humes, L. E., and J. R. Dubno. 2010. {``{Factors affecting speech understanding in older adults}.''} In \emph{The Aging Auditory System}, edited by S. Gordon-Salant, R. D. Frisina, R. R. Fay, and A. N. Popper. New York: Springer-Verlag New York.

\leavevmode\hypertarget{ref-Humes2013}{}%
Humes, Larry E., Gary R. Kidd, and Jennifer J. Lentz. 2013. {``{Auditory and cognitive factors underlying individual differences in aided speech-understanding among older adults}.''} \emph{Frontiers in Systems Neuroscience} 7 (October): 1--16. \url{https://doi.org/10.3389/fnsys.2013.00055}.

\leavevmode\hypertarget{ref-Kidd2012}{}%
Kidd, Gary R., and Larry E. Humes. 2012. {``{Effects of age and hearing loss on the recognition of interrupted words in isolation and in sentences}.''} \emph{The Journal of the Acoustical Society of America} 131 (2): 1434--48. \url{https://doi.org/10.1121/1.3675975}.

\leavevmode\hypertarget{ref-Miller1950}{}%
Miller, George A., and J. C. R. Licklider. 1950. {``{The Intelligibility of Interrupted Speech}.''} \emph{The Journal of the Acoustical Society of America} 22 (2): 167--73. \url{https://doi.org/10.1121/1.1906584}.

\leavevmode\hypertarget{ref-Nasreddine2005}{}%
Nasreddine, Ziad, Natalie Phillips, Valerie Bedirian, Simon Charbonneau, Victor Whitehead, Isabelle Collin, Jeffrey Cummings, and Howard Chertkow. 2005. {``{The Montreal Cognitive Assessment , MoCA : A Brief Screening}.''} \emph{Journal of the American Geriatric Society} 53: 695--99. \url{https://doi.org/10.1111/j.1532-5415.2005.53221.x}.

\leavevmode\hypertarget{ref-Pichora-Fuller2006}{}%
Pichora-Fuller, M. Kathleen, and Gurjit Singh. 2006. {``{Effects of Age on Auditory and Cognitive Processing: Implications for Hearing Aid Fitting and Audiologic Rehabilitation}.''} \emph{Trends in Amplification} 10 (1): 29--59. \url{https://doi.org/10.1177/108471380601000103}.

\leavevmode\hypertarget{ref-Surprenant2001}{}%
Surprenant, Aimée M., and Charles S. Watson. 2001. {``{Individual differences in the processing of speech and nonspeech sounds by normal-hearing listeners}.''} \emph{The Journal of the Acoustical Society of America} 110 (4): 2085--95. \url{https://doi.org/10.1121/1.1404973}.

\end{CSLReferences}


%%%%% REFERENCES

% JEM: Quote for the top of references (just like a chapter quote if you're using them).  Comment to skip.
% \begin{savequote}[8cm]
% The first kind of intellectual and artistic personality belongs to the hedgehogs, the second to the foxes \dots
%   \qauthor{--- Sir Isaiah Berlin \cite{berlin_hedgehog_2013}}
% \end{savequote}

\setlength{\baselineskip}{0pt} % JEM: Single-space References

{\renewcommand*\MakeUppercase[1]{#1}%
\printbibliography[heading=bibintoc,title={\bibtitle}]}

\end{document}
